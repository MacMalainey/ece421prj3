\documentclass{article}
\usepackage[utf8]{inputenc}
\usepackage{minted}
\usepackage{hyperref}

\hypersetup{
  colorlinks=true,
  linkcolor=black,
  filecolor=magenta,      
  urlcolor=blue,
  pdftitle={ECE 421 Project 3 Design},
  pdfpagemode=FullScreen,
}

\title{ECE 421 Project 3 Design}
\author{
  Mackenzie Malainey\\
  \texttt{mmalaine@ualberta.ca}
  \and
  Lora Ma\\
  \texttt{lora@ualberta.ca}
  \and
  Benjamin Kong\\
  \texttt{bkong@ualberta.ca}
}
\date{April 2022}

\begin{document}

\begin{titlepage}
    \maketitle
    \tableofcontents
\end{titlepage}

\section{Server Backend}

\subsection{REST Endpoints}

\subsubsection{User Authentication}

\begin{description}
  \item[POST] \mintinline{text}|/api/v1/user/login/|
  \begin{description}
    \item[Description] \hfill \\
    Logs in a user.
    \item[Request Body Format] \hfill \\
    Form Data
    \item[Request Body Data] \hfill \\
    \mintinline{text}{user_id = USER_PROVIDED_USERNAME} \\
    \mintinline{text}{password = USER_PROVIDED_RAW_PASSWORD}
    \item[Response Cookies] \hfill \\
    ADD \mintinline{text}{user_auth_token}
    \item[Response Status] \hfill \\
    200 - If successful \\
    401 - If \mintinline{text}{user_id} and \mintinline{text}{password} do match an existing user \\
    404 - If request body is malformed % CHECK IF TRUE, IF IT IS WE SHOULD CHANGE THE RETURN TYPE
    \item[Known Issues] 
    \item Attempting to log in to another account when already logged will automatically log out the other user from the server's perspective.
    However this might pose a security concern (especially if user specific data gets cached in the future)
    \item Currently accepts and processes requests without any method of form encryption (dangerously insecure)
  \end{description}

  \item[POST] \mintinline{text}|/api/v1/user/logout/|
  \begin{description}
    \item[Description] \hfill \\
    Logs out the currently logged in user.
    \item[Request Cookies] \hfill \\
    CONTAINS \mintinline{text}{user_auth_token}
    \item[Response Cookies] \hfill \\
    CLEAR \mintinline{text}{user_auth_token}
    \item[Response Status] \hfill \\
    200 - If successful \\
    404 - If header doesn't contain cookie % CHECK IF TRUE, IF IT IS WE SHOULD CHANGE THE RETURN TYPE
    \item[Known Issues] 
    \item Need to verify this removes the cookie in browser
  \end{description}

  \pagebreak
  \item[GET] \mintinline{text}|/api/v1/user/verify/|
  \begin{description}
    \item[Description] \hfill \\
    Verifies that the client's auth cookie matches a known user
    \item[Request Cookies] \hfill \\
    CONTAINS \mintinline{text}{user_auth_token}
    \item[Response Status] \hfill \\
    200 - If successful \\
    404 - If header doesn't contain cookie % CHECK IF TRUE, IF IT IS WE SHOULD CHANGE THE RETURN TYPE
    \item[Response Body Type] \hfill \\
    JSON
    \item[Response Body] \hfill \\
    Username of authenticated user
    \item[Known Issues] 
    \item Need to verify this removes the cookie in browser
  \end{description}
  
  \item[POST] \mintinline{text}|/api/v1/user/register/|
  \begin{description}
    \item[Description] \hfill \\
    Registers a new user.
    \item[Request Body Format] \hfill \\
    Form Data
    \item[Request Body Data] \hfill \\
    \mintinline{text}{user_id = USER_PROVIDED_USERNAME} \\
    \mintinline{text}{password = USER_PROVIDED_RAW_PASSWORD}
    \item[Response Cookies] \hfill \\
    ADD \mintinline{text}{user_auth_token}
    \item[Response Status] \hfill \\
    200 - If \mintinline{text}{user_id} and \mintinline{text}{password} match an existing user \\
    401 - If \mintinline{text}{user_id} and \mintinline{text}{password} do match an existing user \\
    404 - If request body is malformed % CHECK IF TRUE, IF IT IS WE SHOULD CHANGE THE RETURN TYPE
    \item[Response Cookies] \hfill \\
    ADD \mintinline{text}{user_auth_token}
    \item[Response Status] \hfill \\
    200 - If \mintinline{text}{user_id} and \mintinline{text}{password} match an existing user \\
    404 - If header doesn't contain cookie % CHECK IF TRUE, IF IT IS WE SHOULD CHANGE THE RETURN TYPE
    \item[Known Issues] 
    \item Attempting to log in to another account when already logged will automatically log out the other user from the server's perspective.
    However this might pose a security concern (especially if user specific data gets cached in the future)
    \item Currently accepts and processes requests without any method of form encryption (dangerously insecure)
    \item Assumes client enforces proper password requirements
  \end{description}
\end{description}

\pagebreak

\subsubsection{Match Records}

\begin{description}
  \item[POST] \mintinline{text}|/api/v1/user/records/add|
  \begin{description}
    \item[Description] \hfill \\
    Registers match data for the current user
    \item[Request Body Format] \hfill \\
    JSON
    \item[Request Body Data] \hfill
    \begin{minted}{js}
"start_time": START_TIME_IN_SECONDS_FROM_EPOCH_UTC_TIME,
"game_id": {"Connect4", "OttoToot"},
"cpu_level": {"Easy", "Medium", "Hard"},
"duration": DURATION_IN_SECONDS,
"result": {"Win", "Loss", "Tie"}
    \end{minted}
    \item[Request Cookies] \hfill \\
    CONTAINS \mintinline{text}{user_auth_token}
    \item[Response Status] \hfill \\
    200 - If successful \\
    401 - If \mintinline{text}{user_auth_token} does match an existing user \\
    404 - If request body is malformed % CHECK IF TRUE, IF IT IS WE SHOULD CHANGE THE RETURN STATUS
    \item[Known Issues] 
    \item Does not verify \mintinline{js}|"start_time"|.  Probably best to remove this field and use server time to log time upon recording.
  \end{description}

  \item[GET] \mintinline{text}|/api/v1/user/records|
  \begin{description}
    \item[Description] \hfill \\
    Retrieves all the match data from the current user sorted by most recent matches first
    \item[Request Cookies] \hfill \\
    CONTAINS \mintinline{text}{user_auth_token}
    \item[Optional Request Query Parameters] \hfill \\
    \mintinline{text}|limit (default = 10)| \\
    Number of records to return at once \vspace{0.5em} \\
    \mintinline{text}|offset (default = 0)| \\
    Number of records to skip (for pagination) \vspace{0.5em} \\
    \mintinline{text}|before| \\
    Returns matches that happened before (UTC timestamp in seconds) \vspace{0.5em} \\
    \mintinline{text}|after| \\
    Returns matches that happened after (UTC timestamp in seconds) \vspace{0.5em} \\
    \mintinline{text}|sort_by (default = starttime)| \\
    Value to sort by, can be either \mintinline{text}{duration} or \mintinline{text}{starttime} \vspace{0.5em} \\
    \mintinline{text}|asc| \\
    Sort direction, defaults to false unless using \mintinline{text}{sort_by=duration} \vspace{0.5em} \\
    \mintinline{text}|filter| \\
    Only returns elements that match the filter specification (see examples for more info)
    \item[Response Status] \hfill \\
    200 - If successful \\
    404 - If header doesn't contain cookie % CHECK IF TRUE, IF IT IS WE SHOULD CHANGE THE RETURN TYPE
    \item[Response Body Format] \hfill \\
    JSON
    \item[Response Body] \hfill \\
    List of match records
    \item[Known Issues] 
    \item Does not support any form of filtering or sorting other than listed
    \item Handles case where user doesn't exist by returning empty list instead of an error status
    \item Does not apply a maximum or minimum on limit values
    \item Should include a count of how many records there are in total
  \end{description}

  \item[GET] \mintinline{text}|/api/v1/games/records|
  \begin{description}
    \item[Description] \hfill \\
    Retrieves all the match data from the current user sorted by most recent matches first
    \item[Request Cookies] \hfill \\
    CONTAINS \mintinline{text}{user_auth_token}
    \item[Optional Request Query Parameters] \hfill \\
    \mintinline{text}|limit (default = 10)| \\
    Number of records to return at once \vspace{0.5em} \\
    \mintinline{text}|offset (default = 0)| \\
    Number of records to skip (for pagination) \vspace{0.5em} \\
    \mintinline{text}|before| \\
    Returns matches that happened before (UTC timestamp in seconds) \vspace{0.5em} \\
    \mintinline{text}|after| \\
    Returns matches that happened after (UTC timestamp in seconds) \vspace{0.5em} \\
    \mintinline{text}|sort_by (default = starttime)| \\
    Value to sort by, can be either \mintinline{text}{duration} or \mintinline{text}{starttime} \vspace{0.5em} \\
    \mintinline{text}|asc| \\
    Sort direction, defaults to false unless using \mintinline{text}{sort_by=duration} \vspace{0.5em} \\ \\
    \mintinline{text}|filter| \\
    Only returns elements that match the filter specification (see examples for more info)
    \item[Response Status] \hfill \\
    200 - If successful
    \item[Response Body Format] \hfill \\
    JSON
    \item[Response Body] \hfill \\
    List of match records
    \item[Known Issues] 
    \item Does not have max and min values for \mintinline{text}{limit}
    \item Should include a count of how many records there are in total
  \end{description}
\end{description}

\subsubsection{Notes for Future Development}

\begin{itemize}
  \item[] Admin User APIs should be implemented so that a web-created Admin Console may be implemented.
  This was not implemented right now due to time constraints.
  \item[] Check APIs for security vulnerabilities (due to time constaints only minimal security could be implemented)
\end{itemize}

\subsection{Backend Stack}

The backend is implemented using \mintinline{text}|rocket(v0.5.0)| for the backend server framework. 
Through \mintinline{text}|rocket|'s database connection pool library we used \mintinline{text}|diesel|
as the backend database library which interfaces with a \mintinline{text}|sqlite3| database.

The original design was to use \mintinline{text}|rocket(v0.4.4)| with a \mintinline{text}|mongodb| database through \mintinline{text}|rocket|'s database connection pool library.
This would allow us to carry over the prior project's database with minimal issue.
However we found that some of the dependencies had been removed from \mintinline{text}{crates.io} and therefore were not able to use \mintinline{text}|mongodb| with \mintinline{text}|rocket(v0.4.4)|.
Sadly \mintinline{text}|rocket(v0.5.0)| does not support \mintinline{text}|mongodb| and we decided it was best to not homebrew a solution together.  That is what led us to using \mintinline{text}|diesel|
with a \mintinline{text}|sqlite3| database. We felt \mintinline{text}|diesel| was a better option than \mintinline{text}|rusqlite| with its CLI app to be able to create and run database migrations,
embed migrations into the app so that the database could be built on first run as well as the compile time query checking saving a lot of potential headaches during development and for future development and saved on boilerplate code.

\pagebreak

\subsection{Admin CLI}

A local database can be investigated and altered directly using \mintinline{text}|prj3_cli|.
To use, run the CLI and when prompted specify a path for the database you wish to alter.
If no database exists at the given path one will be created.  Then use the next menu to perform various actions on the database.

\section{Web Client}

\subsection{Pages}
In \mintinline{text}{main.rs}, we defined routes to the following pages:
\begin{itemize}
  \item \mintinline{text}{<Homepage/>}: The component for the homepage that routes to \mintinline{text}{/}
  \item \mintinline{text}{<Login/>}: The component for the login page that routes to \mintinline{text}{/login}. Users are able to create an account or log in to their account.
  \item \mintinline{text}{<Leaderboard/>}: The component for the leaderboard page that routes to \mintinline{text}{/leaderboard}. Users are able to view the top ten records for each of the games.
  \item \mintinline{text}{<Connect4/>}: The component for connect 4 that routes to \mintinline{text}{/games/connect4}. It also contains all the initial This page uses the components \mintinline{text}{<PlayScreen/>} and \mintinline{text}{<GameSetup/>}.
  \item \mintinline{text}{<Toot/>}: The component for TOOT and OTTO that routes to \mintinline{text}{/games/toototto}. This page also uses the components \mintinline{text}{<PlayScreen/>} and \mintinline{text}{<GameSetup/>}
\end{itemize}

\subsection{Components}
\begin{itemize}
  \item \mintinline{text}{<GameSetup/>}: The component for the game setup screen for both  \mintinline{text}{<Connect4/>} and \mintinline{text}{<Toot/>}. It uses the component \mintinline{text}{<RadioGroup/>}. This component displays the details of the game and explains how to play. It also gives you that ability to select a difficulty, board size, and disc color. 
  \item \mintinline{text}{<RadioGroup/>}: A components for radio groups such as the radio groups used in \mintinline{text}{<GameSetup/>} where we have the radio groups difficulty, board size, and disc color. 
  \item \mintinline{text}{<PlayScreen/>}: A component for the game. This generates the board and and handles input for the game.
\end{itemize}

\subsection{Design choices for interface components}
There are many choices that exist for interface components:
\begin{itemize}
  \item Color: an important design choice especially when designing for color vision deficiency. By choosing colors carefully and with contrast, we can accommodate for users with color vision deficiency. For example, a common color combination that is difficult for people who experience color blindness is red and green, so designers should avoid using them side by side.
  \item Font: having a readable font is essential, but tactfully using font sizes can be equally important. For example, having headers that are larger and more bold helps the reader recognize the text as being more important to read. Thinner or more small font can be used for descriptions or less important text.
  \item Dimension of windows: users should be able to use our website on monitors of all different sizes and devices of all kinds. For the design to still look appealing across devices, each component should be designed with the variation of dimensions in mind.
  \item Input types: there are a variety of different ways to take in user input and some are more intuitive in certain situations than others. We chose to make choosing the difficulty, board size, and disc color use radio buttons. Radio groups are a good design choice for this situation because only one of the radio buttons should be selected at a time. Similarly, getting text input for the log in/sign up information is a method that is intuitive and familiar to most users. Another input type we used was for the game. When selecting a column to place your piece, the entire column is highlighted when you hover on any cell in that column. This helps indicate to the user that you only have choice over the column and not the specific cell.
\end{itemize} 

\section{Local Setup}

\subsection{Notice}
The tutorial assumes you have the Rust toolchain installed on your system.
Also, the install SQLite3 step is not necessary since sqlite3 is compiled using the library bundled with the \mintinline{text}|libsqlite3-sys| package,
however this does increase the download size for the dependency but for environment consistency is recommended.
Database initialization can be completed by running the CLI or running the server with the features mentioned in the tutorial.

\subsection{Default Config}

The default database path is \mintinline{text}|$PROJ_ROOT$/localdev.db|

The default port that the local web server serves on is \mintinline{text}|8080|

The default port that the backend API server serves on is \mintinline{text}|8000|

IMPORTANT: a proxy is set up for the API calls (see \href{run: ../Trunk.toml}{Trunk.toml}) for the local web server that expects the backend API server to be running on the same machine on the port mentioned above.
If you wish to change the port or host the backend API server on a different device you will need to update the proxy address in the mentioned config file.

\subsection{Instructions}

See the \href{run:../README.md}{readme} for information on setting up the environment and running the server backend and web client.

\section{Key Design Considerations}

\subsection{Computerized Board}

\subsection{Computerized Opponents}

\subsection{Interface Components}

\subsection{Exception Handling}

In a GUI app exception handling must be taken with care.
In a CLI app or script it might be acceptable to simply print out an error message and move on, or just panic on those rare slightly incovnient to safely handle errors.
However, for a GUI app this is much different because we always want to ensuring the app remains responsive and that the user has enough feedback from our app to determine what the next rational step should be.
Since panicking in a GUI app might cause the app to freeze (especially a web app) this should be avoided and any errors that cannot be fixed (such as failed logout attempts) should be reported to the user, so they are aware that something went wrong and can safely determine what the next step should be. 

\subsection{Development Tools}

Development tools are important to have to be able to debug and verify a system outside of the server and client.  We developed an Admin CLI tool to allow for us to create test data or find data in the database to verify the functionality of the server and web app.
The CLI does NOT have support for directly testing the server APIs directly, instead other tools such as \href{https://www.postman.com}{Postman} were used to verify the API endpoints against test data.

\subsection{UI Design Patterns}

\end{document}